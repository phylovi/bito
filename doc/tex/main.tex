\documentclass{article}

\usepackage{amsmath}
\usepackage{amsfonts}
\usepackage{amssymb}
\usepackage{graphicx}
\usepackage[hidelinks]{hyperref}
\usepackage{bm}

\newcommand{\psp}{\slash\!\!\slash}
\newcommand{\qSplit}{q^\textsf{S}\hspace{-1pt}}
\newcommand{\gSplit}{g^\textsf{S}\hspace{-1pt}}
\newcommand{\qPSP}{q^\textsf{PSP}\hspace{-1pt}}
\newcommand{\gPSP}{g^\textsf{PSP}\hspace{-1pt}}

\title{libsbn notes}
\author{Erick}

\begin{document}
%\maketitle


\section*{Setup}
We want to maximize the ELBO of our variational parameterization $q$
\[
L(\bm{\phi},{\bm{\psi}}) := \mathbb{E}_{q_{\bm{\phi},{\bm{\psi}}}(\tau, \bm{\theta})}\log\left(\frac{p(\bm{Y}|\tau, \bm{\theta}) p(\tau, \bm{\theta})}{q_{\bm{\phi}}(\tau)q_{\bm{\psi}}(\bm{\theta}|\tau)}\right) \leq \log p(\bm{Y}).
\]
over $\phi$ and $\psi$.
The discrete part of the variational distribution is given by an SBN distribution $q_{\bm\phi}$, while $q_{\bm\psi}$ is the scalar density.
We will use $g_{\bm{\psi}}(\bm{\epsilon}|\tau)$ to designate the ``reparametrization function'' such that we can get a sample from $q_{\bm\psi}$ by applying $g_{\bm{\psi}}(\bm{\epsilon}|\tau)$ to some normal variates $\bm\epsilon$ to get $\bm\theta$.
We will use component-wise notation $\theta_b = g_{b, \psi}(\bm\epsilon|\tau)$ for branch $b$ of the topology $\tau$.
So, applying the reparametrization trick we get
\begin{equation}
L(\bm{\phi},{\bm{\psi}}) = \mathbb{E}_{
    q_{\bm{\phi}}(\tau,\bm{\epsilon})}
    \log\left(
        \frac
        {p(\bm{Y}|\tau,g_{\bm{\psi}}(\bm{\epsilon}|\tau))p(\tau, g_{\bm{\psi}}(\bm{\epsilon}|\tau))}
        {q_{\bm{\phi}}(\tau)q_{\bm{\psi}}(g_{\bm{\psi}}(\bm{\epsilon}|\tau)|\tau)}
    \right).
\label{eq:L}
\end{equation}

\section*{Gradient with respect to SBN parameters $\phi$}

TODO

\section*{Gradient with respect to scalar model parameters $\psi$}

Taking the derivative of the numerator of \eqref{eq:L} with respect to a $\psi_i$ gives

\begin{equation}
    \sum_b
    \left(
        \frac{\partial \log p(Y | \tau, \bm\theta)}{\partial \theta_b}
        +
        \frac{\partial \log p(\tau, \bm\theta)}{\partial \theta_b}
    \right)
    \frac{\partial g_{b,\bm\psi}(\bm\epsilon | \tau)}{\partial \psi_i}
    \label{eq:dLdPsi}
\end{equation}
Given that $p(\tau, \bm\theta) = p(\tau) p(\bm\theta | \tau)$ this simplifies to
\begin{equation}
    \sum_b
    \left(
        \frac{\partial \log p(Y | \tau, \bm\theta)}{\partial \theta_b}
        +
        \frac{\partial \log p(\bm\theta | \tau)}{\partial \theta_b}
    \right)
    \frac{\partial g_{b,\bm\psi}(\bm\epsilon | \tau)}{\partial \psi_i}
    \label{eq:dPdPsi}
\end{equation}
The left hand term of the sum is the standard log phylogenetic gradient, while the right hand term is the derivative of the log branch length prior with respect to a specific branch.

In the case where the branch length prior $p(\bm\theta | \tau)$ is a product of terms $\prod_b p_b(\theta_b)$ this becomes very simple:
\[
    \frac{\partial \log p(\bm\theta | \tau)}{\partial \theta_b} =
    \frac{\partial \log p_b(\theta_b)}{\partial \theta_b}.
\]

For the denominator we only need to worry about the $q_{\bm\psi}$ term.
We will assume that $q_{\bm\psi}(\bm\theta | \tau)$ factors as a product of terms across branches of the topology, so
\[
    - \log q_{\bm{\psi}}(g_{\bm{\psi}}(\bm{\epsilon}|\tau)|\tau) =
    - \sum_b \log q_{b, \bm{\psi}}(g_{b, \bm{\psi}}(\bm{\epsilon}|\tau)|\tau).
\]

So, using \eqref{eq:dPdPsi} we get the derivative of $L(\bm\phi, \bm\psi)$ with respect to $\psi_i$ being the sum over $b$ of:
\begin{equation*}
    \left(
        \frac{\partial \log p(Y | \tau, \bm\theta)}{\partial \theta_b}
        +
        \frac{\partial \log p(\bm\theta | \tau)}{\partial \theta_b}
    \right)
    \frac{\partial g_{b,\bm\psi}(\bm\epsilon | \tau)}{\partial \psi_i}
    - \frac{\log q_{b, \bm\psi}(g_{b, \bm\psi}(\bm\epsilon|\tau)|\tau)}{\partial \psi_i}.
\end{equation*}

\subsubsection*{Split-based parametrization:}
First consider the split-based parametrization given in the ``simple independent approximation'' section of the 2018 ICLR paper.
If we use $b \slash \tau$ to mean the split given by branch $b$ of topology $\tau$, then the parametrization is
\[
q_{b, \bm{\psi}}(\theta_b | \tau) := \qSplit(\theta_b; \psi_{b \slash \tau})
\qquad
g_{b,\bm\psi}(\bm\epsilon | \tau) := \gSplit(\epsilon_b; \psi_{b \slash \tau}).
\]
for some collection of functions $q(\theta; \psi)$ with corresponding $g(\epsilon; \psi)$.

To implement this, index $\bm\psi$ by splits.
To get the gradient of $L(\bm\phi, \bm\psi)$ with respect to $\bm\psi$, we can start with a zero vector then loop over the branches $b$, and for each one incrementing the $b \slash \tau$ entry by
\begin{equation*}
    \left(
        \frac{\partial \log p(Y | \tau, \bm\theta)}{\partial \theta_b}
        +
        \frac{\partial \log p(\bm\theta | \tau)}{\partial \theta_b}
    \right)
    \frac{\partial \gSplit(\epsilon_b; \psi_{b \slash \tau})}{\partial \psi_{b \slash \tau}}
    - \frac{\log \qSplit(\gSplit(\epsilon_b; \psi_{b \slash \tau}); \psi_{b \slash \tau})}{\partial \psi_{b \slash \tau}}.
\end{equation*}


\subsubsection*{Primary subsplit pair parametrization:}
Next consider the primary subsplit pair parametrization.
If we use $b \psp \tau$ to mean the primary subsplit pair given by branch $b$ of topology $\tau$, then the parametrization is formally very similar:
% TODO what is this bold epsilon?
\[
q_{b, \bm{\psi}}(\theta_b | \tau) := \qPSP(\theta_b; \bm\psi_{b \psp \tau})
\qquad
g_{b,\psi}(\bm\epsilon | \tau) := \gPSP(\bm\epsilon_{b \psp \tau}; \bm\psi_{b \psp \tau}).
\]
Note that $\bm\psi_{b \psp \tau}$ is a vector of the entries of $\bm\psi$ for the three edges contributing to the PSP.

In the PSP parametrization, the parameters of the approximating distribution for a PSP $b \psp \tau$ are the sum of the split corresponding to the split $b \slash \tau$ and the two primary subsplits refining $b \slash \tau$.

\subsection*{Families of scalar variational distributions}
\subsubsection*{Log-normal:}
If $q$ is log-normal with location $\mu$ and scale $\sigma$,
\[
\log q_{\varsigma, \psi}(\theta) := c - \log \theta - \log \sigma_\varsigma - \frac{(\log \theta - \mu_\varsigma)^2}{2 \sigma_\varsigma^2}.
\]
for a constant $c$.
Because taking $g_{\varsigma,\psi}(\epsilon) := \exp(\mu_\varsigma + \sigma_\varsigma \epsilon)$,
\begin{align*}
\log q_{\varsigma, \psi}(g_{\varsigma,\psi}(\epsilon))
& := c - (\mu_\varsigma + \sigma_\varsigma \epsilon)
    - \log \sigma_\varsigma
    - \frac{(\mu_\varsigma + \sigma_\varsigma \epsilon - \mu_\varsigma)^2}{2 \sigma_\varsigma^2} \\
 & := c - \mu_\varsigma - \sigma_\varsigma \epsilon - \log \sigma_\varsigma - \frac{1}{2}
\end{align*}
Thus,
\begin{equation}
    \frac{\partial \log q_{\varsigma, \psi}(g_{\varsigma,\psi}(\epsilon))}{\partial \mu_\varsigma} = -1
    \qquad
    \frac{\partial \log q_{\varsigma, \psi}(g_{\varsigma,\psi}(\epsilon))}{\partial \mu_\varsigma} = -\epsilon - \frac{1}{\sigma_\varsigma}.
\end{equation}
Also,
\begin{equation}
    \frac{\partial g_{\varsigma,\psi}(\epsilon)}{\partial \mu_\varsigma} = g_{\varsigma,\psi}(\epsilon)
    \qquad
    \frac{\partial g_{\varsigma,\psi}(\epsilon)}{\partial \sigma_\varsigma} = g_{\varsigma,\psi}(\epsilon) \cdot \epsilon \, .
\end{equation}

% \begin{figure}[h]
% \centering
% \includegraphics[width=0.35\textwidth]{figures/subsplit.pdf}
% \caption{\
% A subsplit structure.
% }
% \label{fig:subsplit}
% \end{figure}

\nocite{vbpi}

\bibliographystyle{plain}
\bibliography{main}

\end{document}
